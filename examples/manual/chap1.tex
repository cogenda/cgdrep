\chapter{Physics in Genius Device Simulator}

Since Gummel's original work, the drift-diffusion model has been widely used in semiconductor device simulation.  It is now the de facto industry standard in the field.

The original DD model can be achieved by following approximation from hydrodynamic model:
\begin{itemize}
\item Light speed is much faster than carrier speed.
\item All the collisions are elastic.
\item Band-gap does not change during collisions.
\item Carrier temperature equals to lattice temperature and keeps equilibrium.
\item The gradient of driving force should keep small.
\item Carrier degeneration can be neglected.
\end{itemize}

Some improvements have been applied to DD model for extend its capability.  These "patches" of course make things complex, but they can deal with realistic problems.

This chapter describes the DD model and its variations used in GENIUS code for describing semiconductor device behavior as well as physical based parameters such as mobility, recombination rate and so on.

\section{Level 1 Drift-Diffusion Equation}
\label{sec:Equation:DDML1}
Level 1 Drift-Diffusion ($\mathbf{DDML1}$) \index{DDML1!Equation}is the fundamental solver of GENIUS code for that lattice temperature keeps constant throughout the solving procedure.

The primary function of $\mathbf{DDML1}$ is to solve the following set of partial differential equations, namely Poisson's equation, along with the hole and electron continuity equations:

\marginhead{Poisson's Equation}
\begin{equation}
\nabla \cdot \varepsilon \nabla \psi = - q\left( p - n + {N_D}^+ - {N_A}^- \right)
\end{equation}
\par
\par
\index{Poisson's Equation}where, $\psi$ is the electrostatic potential of the vacuum level. This choice makes the descriptions of metal-oxide-semiconductor contact and heterojunction easier. $n$ and $p$ are the electron and hole concentration, ${N_D}^{+}$ and ${N_A}^{-}$ are the ionized impurity concentrations. $q$ is the magnitude of the charge of an electron.

The relationship between conduct band $E_c$, valence band $E_v$ and vacuum level $\psi$ is:
\begin{equation}\begin{array}{l}
 E_c  =-q\psi-\chi-\Delta E_c \\
 E_v  =E_c-E_g+\Delta E_v.
\end{array}\end{equation}

Here, $\chi$ is the electron affinity. $E_g$ is the bandgap of semiconductor. $\Delta E_c$ and $\Delta E_v$ are the bandgap shift caused by heavy doping or mechanical strain.

Furthermore, the relationship between vacuum level $\psi$ and intrinsic Fermi potential $\psi_ {\rm intrinsic}$ is:
\begin{equation}
\psi = \psi _{\rm intrinsic} - \frac{\chi}{q} - \frac{E_g}{2q} - \frac{k_b T}{2q}\ln \left(\frac{N_c}{N_v} \right)
\end{equation}

The reference 0\si\eV of energy is set to intrinsic Fermi level of equilibrium state in GENIUS code.

\marginhead{Continuity Equations}\index{Continuity equation!DDML1}The continuity equations for electrons and holes are defined as follows:

\begin{equation}\begin{array}{l}
\displaystyle \frac{\partial n}{\partial t}  = \frac{1}{q}\nabla \cdot \vec{J}_n - (U - G) \\
\displaystyle \frac{\partial p}{\partial t}  = - \frac{1}{q}\nabla \cdot \vec{J}_p - (U - G)
\end{array}\end{equation}

where $\vec{J}_n$ and $\vec{J}_p$ are the electron and hole current densities, $U$ and $G$ are the recombination and generation rates for both electrons and holes.

\marginhead{Drift-Diffusion Current Equations}\index{Drift-diffusion current!DDML1}The current densities $\vec{J}_n$ and $\vec{J}_p$ are expressed in terms of the level 1 drift-diffusion model here.

\begin{equation} \label{eq:Equation:DDML1:DDMCurrent}
\begin{array}{l}
 \vec{J}_n  = q\mu_n n \vec{E}_n + q D_n \nabla n \\
 \vec{J}_p  = q\mu_p p \vec{E}_p - q D_p \nabla p
\end{array}\end{equation}
where $\mu_n$ and $\mu_p$ are the electron and hole mobilities. $D_n=\frac{k_bT}{q}\mu_n$ and $D_p=\frac{k_bT}{q}\mu_p$ are the electron and hole diffusivities, according to Einstein relationship.

\marginhead{Effective Electrical Field}$\vec{E}_n$ and $\vec{E}_p$ are the effective driving electrical field to electrons and holes, which related to local band diagram. The band structure of heterojunction has been taken into account here \cite{Lindefelt1994}.

\begin{equation}\label{eq:Equation:DDML1:DrivingField}\begin{array}{l}
\displaystyle \vec{E}_n  = \frac{1}{q}\nabla E_c - \frac{k_b T}{q}\nabla \left( \ln (N_c ) - \ln (T^{3/2} ) \right) \\
\displaystyle \vec{E}_p  = \frac{1}{q}\nabla E_v + \frac{k_b T}{q}\nabla \left( \ln (N_v ) - \ln (T^{3/2} ) \right)
\end{array}\end{equation}
The lattice temperature keeps uniform throughout $\mathbf{DDML1}$, the above temperature gradient item takes no effect in fact.

By substituting drift-diffusion model into the current density expressions, and combining with Poisson's equation, the following basic equations for $\mathbf{DDML1}$ are obtained:

\begin{equation}\begin{array}{l}
\displaystyle \frac{\partial n}{\partial t}  = \nabla \cdot \left (\mu_n n \vec{E}_n + \mu_n\frac{k_b T}{q}\nabla n \right) - (U - G) \\
\displaystyle \frac{\partial p}{\partial t}  = -\nabla \cdot \left (\mu_p p\vec{E}_p - \mu_p\frac{k_b T}{q}\nabla p \right) - (U - G)  \\
\displaystyle \nabla \cdot \varepsilon \nabla \psi  = - q(p - n + {N_D}^{+} - {N_A}^{-} )
\end{array}\end{equation}

$\mathbf{DDML1}$ is suitable for PN diode, BJT transistor and long gate MOSFET simulation. It is robust, and runs pretty fast for real work. The detailed discretization scheme can be found at [[TODO]].

\section{Level 2 Drift-Diffusion Equation} \label{sec:Equation:DDML2}\index{DDML2!Equation}
The Level 2 DD model considers the influence of lattice temperature by solving the extra thermal equation simultaneously with the electrical equations. Also, the formula of drift-diffusion equation should be modified according to \cite{Selberherr1984}.

\index{Drift-diffusion current!DDML2}The electron diffusion current in $\mathbf{DDML1}$ can be written as:
\begin{equation}
{\vec{J}}_{n,{\rm diff}} = \frac{k_b T}{q}\mu_n q\nabla n = k_b T\mu_n \nabla n
\end{equation}

\marginhead{Temperature Gradient Correction} But for DDML2, it has the form of
\begin{equation}
{\vec{J}}_{n,{\rm diff}} = \mu_n k_b (T\nabla n + n\nabla T)
\end{equation}

The hole diffusion current should be modified in the same manner.
\begin{equation}
{\vec{J}}_{p,{\rm diff}} = -\mu_p k_b (T\nabla p + p\nabla T)
\end{equation}

\marginhead{Heat Flow Equation}\index{Heat flow equation!DDML2}The following heat flow equation is used:
\begin{equation}
\rho c_p \frac{\partial T}{\partial t} = \nabla \cdot \kappa \nabla T + \vec{J} \cdot \vec{E} + (E_g + 3k_b T) \cdot (U - G)
\end{equation}
where $\rho$ is the mass density of semiconductor material. $c_p$ is the heat capacity. $\kappa$ is the thermal conductivity of the material. $\vec{J}\cdot\vec{E}$ is the joule heating of the current.  $(E_g + 3k_b T) \cdot (U - G)$ is the lattice heating due to carrier recombination and generation.

From the discussions above, the governing equations for DDML2 are as follows:
\begin{equation}\begin{array}{l}
\displaystyle \frac{\partial n}{\partial t}  = \nabla \cdot \left(\mu_n n \vec{E}_n + \mu_n \frac{k_b T}{q}\nabla n + \mu_n \frac{k_b \nabla T}{q} n\right) - \left( {U - G} \right) \\
\displaystyle \frac{\partial p}{\partial t}  = -\nabla \cdot \left(\mu_p p \vec{E}_p - \mu_p \frac{k_b T}{q}\nabla p - \mu_p \frac{k_b \nabla T}{q} p\right) - \left( {U - G} \right) \\
\displaystyle \nabla \cdot \varepsilon \nabla \psi  = - q\left( {p - n + {N_D}^+ - {N_A}^ - } \right)  \\
\displaystyle \rho c_p \frac{\partial T}{\partial t}  = \nabla \cdot \kappa \nabla T + \vec{J} \cdot \vec{E} + (E_g + 3k_b T) \cdot (U - G)
\end{array}\end{equation}
This model can be used as power transistor simulation as well as breakdown simulation. Unfortunately, nearly all the physical parameters are related with temperature. They should be considered during self consistent simulation, which greatly slows down the speed. The $\mathbf{DDML2}$ solver runs 50-70\% slower than $\mathbf{DDML1}$. However, it seems no convergence degradation happens in most of the case. The discretization scheme can be found at [[TODO]].

\section{Level 3 Energy Balance Equation}\label{sec:Equation:EBML3} \index{EBML3!Equation}
Energy Balance (EB) Model \cite{PISCES-2ET} is introduced into GENIUS code for simulating short channel MOSFET. This is a simplification of full hydrodynamic (HD) model \cite{Aste2003}. The current density expressions from the drift-diffusion model are modified to include additional coupling to the carrier temperature. Also, reduced carrier energy conservation equations, which derived from second order moment of Boltzmann Transport Equation, are solved consistently with drift-diffusion model. The simplification from HD to EB makes sophisticated Scharfetter-Gummel discretization, which ensures stability, can still be used in numerical solution.

\marginhead{Current Equation for EBM}\index{Drift-diffusion current!EBML3}
The current density $\vec{J}_n$ and $\vec{J}_p$ are then expressed as:
\begin{equation}\begin{array}{l}
 \vec{J}_n  = q\mu_n n \vec{E}_n + k_b \mu_n \left( {n\nabla T_n + T_n \nabla n} \right)  \\
 \vec{J}_p  = q\mu_p p \vec{E}_p - k_b \mu_p \left( {p\nabla T_p + T_p \nabla p} \right)
\end{array}\end{equation}
where, $T\_n$ and $T\_p$ are electron and hole temperature, respectively. The difference between the equations above and the carrier density equations in $\mathbf{DDML2}$ is that lattice temperature is replaced by carrier temperature.

\marginhead{Energy Balance Equations}\index{Energy-balance equation!EBML3}
In addition, the energy balance model includes the following electron and hole energy balance equations:
\begin{equation}\begin{array}{l}
\displaystyle \frac{\partial \left( {n\omega _n } \right)} {\partial t} + \nabla \cdot \vec{S}_n  = \vec{E}_n \cdot \vec{J}_n +H_n \\
\displaystyle \frac{\partial \left( {p\omega _p } \right)} {\partial t} + \nabla \cdot \vec{S}_p = \vec{E}_p \cdot \vec{J}_p +H_p
\end{array}\end{equation}
where, $\omega _n$ and $\omega _p$ are electron and hole energy. For HD model, the carrier energy includes thermal and kinetic terms $\omega _c =\frac{3}{2}k_bT_c + \frac{1}{2}m^* v_c^2$, but only thermal energy for EB model $\omega _c = \frac{3}{2}k_bT_c$. Here $c$ stands for $n$ or $p$. $\omega_0=\frac{3}{2}k_bT$ is the carrier equilibrium energy, for carrier temperature equals to lattice temperature.

$\vec{S}_n$ and $\vec{S}_p$ are the flux of energy:

\begin{equation}\begin{array}{l}
\displaystyle \vec{S}_n  = - \kappa _n \nabla T_n - \left( \omega_n + k_b T_n \right) \frac{\vec{J}_n} {q}  \\
\displaystyle \vec{S}_p  = - \kappa _p \nabla T_p + \left( \omega_p + k_b T_p \right) \frac{\vec{J}_p} {q}
\end{array}\end{equation}

The heat conductivity parameter for carriers can be expressed as:
\begin{equation}
\kappa_c=(\frac{2}{5}+\gamma)\frac{{k_b}^2}{q}T_c\mu_cc
\end{equation}
where $c$ stands for $n$ and $p$, respectively. The constant parameter $\gamma$ equals $-0.7$ in the GENIUS code.
\par
The $H_n$ and $H_p$ are the rate of net loss of carrier kinetic energy:
\par
\begin{widetext}
\begin{subequations}
\begin{align}
 H_n =  \left( R_{\rm Aug} - G \right) \cdot \left( E_g + \frac{3k_b T_p } {2} \right) -
        \frac{3k_b T_n } {2}\left( R_{\rm SHR} + R_{\rm dir} - G \right) 
  -\frac{{n\left( {\omega _n - \omega _0 } \right)}}{{\tau _n }} \\
 H_p =  \left( R_{\rm Aug} - G \right) \cdot \left( E_g + \frac{3k_b T_n } {2} \right) -
        \frac{3k_b T_p}{2}\left( R_{\rm SHR} + R_{\rm dir} - G \right) 
  - \frac{p\left( \omega _p - \omega _0 \right)}{\tau _p }
\end{align}
\end{subequations}
\end{widetext}
\marginhead{Lattice Heat Equation for EBM}\index{Heat flow equation!EBML3}At last, the lattice heat flow equation should be rewritten as:
\begin{equation}
\rho c_p \frac{\partial T}{\partial t} = \nabla \cdot \kappa \nabla T + H
\end{equation}
where

\begin{equation}
H = R_{\rm SHR} \cdot \left( E_g + \frac{3k_b T_p } {2} + \frac{3k_b T_n } {2} \right) + \frac{n\left( \omega _n - \omega _0 \right)} {\tau _n } + \frac{p\left( \omega _p - \omega _0 \right)}{\tau _p}
\end{equation}
The carrier energy is mainly contributed by joule heating term $\vec{E}_c\cdot
      \vec{J}_c$, and heating (cooling) due to carrier generation (recombination) term. The carriers exchange
      energy with lattice by collision, which described by energy relaxation term
$\tau_{\omega
      _c}$. This model is suitable for simulations of sub-micron MOS (channel length $1\sim 0.1$\si{\micro\meter}) and advanced BJT.  However, the computation burden of EB method is much higher than
      DD. And the convergence of EB solver is difficult to achieve, which requires stricter initial values and a more powerful inner linear solver. The discretization scheme can be found at [[TODO]].
\par
From the above discussion, all governing equations of DD/EB method is elliptical or parabolic. From mathematical point of view, not like hyperbolic system\footnote{One have to face discontinuous problem, i.e. shock wave.}, the solution of elliptical or parabolic system is always smooth. The numerical techniques required for these systems are simple and mature.  As a result, DD/EB method is preferred against full hydrodynamic method.
\par

\section{Band Structure Model}
The band structure parameters, including bandgap $E_g$, effective density of
      states in the conduction band $N_c$ and valence band $N_v$, and intrinsic carrier concentration $n_{ie}$, are the most important and fundamental physical parameters for semiconductor material\cite[Sze1981]{}.

\marginhead{Effective Density of States} Effective density of states\index{Density of states!effective} in conduction and valence bands are defined as follows:

\begin{subequations}
\begin{align}
 N_{c}  \equiv 2\left( \frac{{m_{n}}^{*}k_{b}T}{2\pi\hbar^2}\right)^{3/2}\\
 N_{v}  \equiv 2\left( \frac{{m_{p}}^{*}k_{b}T}{2\pi\hbar^2}\right)^{3/2}
\end{align}
\end{subequations}
The temperature dependencies of effective density of states is fairly simple:
\par
\begin{subequations}
\begin{align}
 N_{c}\left(T\right) = N_c \left( 300\si{kelvin} \right)\left( \frac{T}{300\si{\kelvin}}\right)^{1.5}\\
 N_{v}\left(T\right)  = N_v \left( 300\si{\kelvin} \right)\left( \frac{T}{300\si{\kelvin}}\right)^{1.5}
\end{align}
\end{subequations}
\marginhead{Bandgap}The bandgap in GENIUS is expressed as follows:
\par
\par
\begin{subequations}
\begin{align}
 E_g (T) = E_g (0) - \frac{\alpha T^{\rm 2} }{T + \beta} \\
  =E_g (300) + \alpha\left[ \frac{300^{\rm 2} }{300 + \beta} - \frac{T^2}{T + \beta} \right]
\end{align}
\end{subequations}
\marginhead{Bandgap Narrowing due to Heavy Doping}When bandgap narrowing effects\index{Bandgap narrowing!Slotboom model}
due to heavy doping takes place \cite[Slotboom1977]{}, the band edge shifts:
\par
\par
\begin{equation}
\Delta E_g = \frac{E_{\rm bgn}}{2k_b T}\left[ \ln \frac{N_{\rm total}}{N_{\rm ref}} + \sqrt {\left(
        \ln \frac{N_{\rm total}}{N_{\rm ref}} \right)^2 + C_{\rm bgn}} \right].
\end{equation}
The intrinsic concentration should be modified:
\par
\begin{equation}
n_{ie}=\sqrt{N_c N_v } \exp\left(-\frac{E_g}{2 k_b T} \right) \cdot \exp(\Delta E_g)
\end{equation}
Since the carrier current \eqref{eq:Equation:DDML1:DrivingField},
p. \pageref{eq:Equation:DDML1:DrivingField} involves the energy level of
      conduction band $N_{c}$ and valence band $N_{v}$, the
      bandgap shift should be attributed to them. The bandgap narrowing is attributed half to the conduction band and
      another half to the valence band as default:
\par
\begin{subequations}
\begin{align}
 E_c'  =E_c-\frac{1}{2}\Delta E_g \\
 E_v'  =E_v+\frac{1}{2}\Delta E_g
\end{align}
\end{subequations}
The parameters used in the default band structure model is listed in
\ref{tab:Equation:Band:Default:Param}, p. \pageref{tab:Equation:Band:Default:Param}.
\par

\begin{wtable}{lllll}
\caption{\label{tab:Equation:Band:Default:Param}Parameters of the Default band structure model}\\
\toprule
 Symbol
& Parameter
& Unit
& Silicon
& GaAs\\
\hline
$E_g(300)$
& $\mathbf{EG300}$
& \si{\eV}
& 1.1241
& 1.424
\\
 $\alpha$
& $\mathbf{EGALPH}$
& \si{\eV\per\kelvin}
& $2.73\times10^{-4}$
& $5.405\times10^{-4}$
\\
 $\beta$
& $\mathbf{EGBETA}$
& \si{\kelvin}
& $0$
& $204$
\\
 $E_{\rm bgn}$
& $\mathbf{V0.BGN}$
& \si{\eV}
& $6.92\times10^{-3}$
& 0
\\
 $N_{\rm ref}$
& $\mathbf{N0.BGN}$
& \si{\cubic\centi\meter}
& $1.30\times10^{17}$
& $1\times10^{17}$
\\
 $C_{\rm bgn}$
& $\mathbf{CON.BGN}$
& -
& $0.5$
& $0.5$
\\
 $m_n$
& $\mathbf{ELECMASS}$
& $m_0$
& $1.0903$
& $0.067$
\\
 $m_p$
& $\mathbf{HOLEMASS}$
& $m_0$
& $1.1525$
& $0.6415$
\\
 $N_c(300)$
& $\mathbf{NC300}$
& \si{\cubic\centi\meter}
& $2.86\times10^{19}$
& $4.7\times10^{17}$
\\
 $N_v(300)$
& $\mathbf{NV300}$
& \si{\cubic\centi\meter}
& $3.10\times10^{19}$
& $7.0\times10^{18}$\\
\bottomrule
\end{wtable}

\subsection{Band structure of compound semiconductors}
[[TODO]]
\par
\subsubsection{Band Structure of SiGe}
[[TODO]]
\par
\subsubsection{Band Structure of Tertiary Compound Semiconductor}
[[TODO]]
\par
\marginhead{Bandgap}[[TODO]]
\par
\par
\marginhead{Electron Affinity}[[TODO]]
\par
\par
\marginhead{Effective Mass}[[TODO]]
\par
\par
\marginhead{Density of States}[[TODO]]
\par
\par
\subsection{Schenk's Bandgap Narrowing Model}
[[TODO]] Equations of Schenk's model
\par
The Schenk's bandgap narrowing model is available for silicon, and can be loaded with the option
$\mathbf{Schenk}$ in the $\mathbf{PMI}$ command.
\par
\section{Carrier Recombination}
Three recombination mechanisms are considered in GENIUS at present, including Shockley-Read-Hall, Auger, and
      direct (or radiative) recombination. The total recombination is considered as the sum of all:
\par
\begin{equation}
U = U_n = U_p = U_{\rm SRH} + U_{\rm dir} + U_{\rm Auger}
\end{equation}
where $U_{\rm SRH}$, $U_{\rm dir}$ and $U_{\rm Auger}$ are SRH recombination, direct recombination and Auger recombination,
      respectively.
\par
\marginhead{SRH Recombination}Shockley-Read-Hall (SRH) recombination\index{SRH recombination} see index ''Shockley-Read-Hall recombination''
rate is determined by the following formula:
\par
\par
\begin{equation}
U_{\rm SRH}=\dfrac{pn-{n_{ie}}^2}{\tau_p\left[n+n_{ie}\exp\left(\dfrac{\bf
        ETRAP}{kT_L}\right)\right]+\tau_n\left[p+n_{ie}\exp\left(\dfrac{ -{\bf ETRAP}}{kT_L}\right)\right]}
\end{equation}
where $\tau_n$ and $\tau_p$ are carrier life
      time\index{Carrier life-time}, which dependent on impurity concentration
\cite[Roulston1982]{}.
\par
\begin{subequations}
\begin{align}
 \tau_n  =\frac{{\bf TAUN0}}{1+N_{\rm total}/ {\bf NSRHN}} \\
 \tau_p  =\frac{{\bf TAUP0}}{1+N_{\rm total}/ {\bf NSRHP}}
\end{align}
\end{subequations}
\marginhead{Auger Recombination}The Auger recombination\index{Auger recombination}
is a three-carrier recombination process, involving either two electrons and one hole or two
        holes and one electron. This mechanism becomes important when carrier concentration is large.
\par
\par
\begin{equation}
U_{\rm Auger}={\bf AUGN} \left(pn^2-n{n_{ie}}^2 \right)+{\bf AUGP}(np^2-p{{n_{ie}}^2})
\end{equation}
where $\mathbf{AUGN}$ and $\mathbf{AUGP}$ are Auger coefficient for electrons and
      holes. The value of Auger recombination $U_{\rm Auger}$ can be negative some times,
      which refers to Auger generation.
\par
\marginhead{Direct Recombination}The direct recombination\index{Direct recombination}
model expresses the recombination rate as a function of the carrier concentrations
$n$ and $p$, and the effective intrinsic density $n_{ie}$:
\par
\par
\begin{equation}
U_{\rm dir}={\bf DIRECT}(np-{n_{ie}}^2)
\end{equation}
The default value of the recombination parameters are listed in Tab.~\ref{tab:Equation:Recomb:Param}, p. \pageref{tab:Equation:Recomb:Param}:
\par

\begin{wtable}{lllll}
\caption{\label{tab:Equation:Recomb:Param}Default values of recombination parameters} \\
\toprule
 Parameter
& Unit
& Silicon
& GaAs
& Ge\\
\hline
$\mathbf{ETRAP}$
& \si{\eV}
& 0
& 0
& 0
\\
 $\mathbf{DIRECT}$
& $cm^3s^{-1}$
& 1.1e-14
& 7.2e-10
& 6.41e-14
\\
 $\mathbf{AUGN}$
& $cm^6s^{-1}$
& 1.1e-30
& 1e-30
& 1e-30
\\
 $\mathbf{AUGP}$
& $cm^6s^{-1}$
& 0.3e-30
& 1e-29
& 1e-30
\\
 $\mathbf{TAUN0}$
& \si{\second}
& 1e-7
& 5e-9
& 1e-7
\\
 $\mathbf{TAUP0}$
& \si{\second}
& 1e-7
& 3e-6
& 1e-7
\\
 $\mathbf{NSRHN}$
& $cm^3$
& 5e16
& 5e17
& 5e16
\\
 $\mathbf{NSRHP}$
& $cm^3$
& 5e16
& 5e17
& 5e16\\
\bottomrule
\end{wtable}

\marginhead{Surface Recombination}At semiconductor-insulator interfaces, additional SRH recombination can be introduced. The surface
        recombination rate has the unit $cm^{-2}s{-1}$, and is calculated
        with
\par
\par
\begin{equation}
U_{\rm Surf}=\dfrac{pn-{n_{ie}}^2}{\dfrac{1}{\bf STAUN}\left(n+n_{ie}\right)+\dfrac{1}{\bf
        STAUP}\left(p+n_{ie}\right)}.
\end{equation}
The surface recombination velocities, $\mathbf{STAUN}$ and $\mathbf{STAUP}$, have the unit of \si{\centi\meter\per\second}, and the default value of 0.
\par
\section{Mobility Models}
\label{sec:Equation:Mobility}
Carrier mobility is one of the most important parameters in the carrier transport model. The DD model
      itself, developed at early 1980s, is still being used today due to advanced mobility model enlarged its ability to
      sub-micron device.
\par
Mobility modeling is normally divided into: low field behavior, high field behavior and mobility in the
      (MOS) inversion layer.
\par
The low electric field behavior has carriers almost in equilibrium with the lattice. The low-field mobility
      is commonly denoted by the symbol $\mu_{n0}$, $\mu_{p0}$.
      The value of this mobility is dependent upon phonon and impurity scattering. Both of which act to decrease the low
      field mobility. Since scattering mechanism is depended on lattice temperature, the low-field mobility is also a
      function of lattice temperature.
\par
The high electric field behavior shows that the carrier mobility declines with electric field because the
      carriers that gain energy can take part in a wider range of scattering processes. The mean drift velocity no
      longer increases linearly with increasing electric field, but rises more slowly. Eventually, the velocity doesn't
      increase any more with increasing field but saturates at a constant velocity. This constant velocity is commonly
      denoted by the symbol $v_{sat}$. Impurity scattering is relatively insignificant for
      energetic carriers, and so $v_{sat}$ is primarily a function of the lattice
      temperature.
\par
Modeling carrier mobilities in inversion layers introduces additional complications. Carriers in inversion
      layers are subject to surface scattering, carrier-carrier scattering, velocity overshoot and quantum mechanical
      size quantization effects. These effects must be accounted for in order to perform accurate simulation of MOS
      devices. The transverse electric field is often used as a parameter that indicates the strength of inversion layer
      phenomena.
\par
It can be seen that some physical mechanisms such as velocity overshoot and quantum effect which can't be
      described by DD method at all, can be taken into account by comprehensive mobility model. The comprehensive
      mobility model extends the application range of DD method. However, when the EB method (which accounts for
      velocity overshoot) and QDD method (including quantum effect) are used, more calibrations are needed to existing
      mobility models.
\par
\subsection{Bulk Mobility Models}
The first family of mobility models were designed to model the carrier transport at low electric fields.
        They usually focus on the temperature and doping concentration dependence of the carrier mobilities. The
        surface-related or transverse E-field effects are \emph{not}
included in these models. On the
        other hand, in GENIUS, these low-field mobilities models are coupled to a velocity saturation model to account
        for the carrier velocity saturation effect. This family of mobility models are suitable for bulk device, such as
        bipolar transistors.
\par
In brief, the low field carrier mobility is first computed, then a velocity saturation formula is applied
        to yield the corrected mobility value. Three choices are available for the low-field mobility calculation, each
        described in one of the following sub-sections. The choices of velocity saturation is described in the last
        sub-section.
\par
\subsubsection{Analytic Mobility Model}
\label{sec:Equation:Mobility:Bulk:Analytic}
\index{mobility!Analytic model}In the GENIUS code, the Analytic Mobility model
\cite[Selberherr1984P]{} is the
          default low field mobility model for all the material. It is an concentration and temperature dependent
          empirical mobility model expressed as:
\par
\begin{equation}
\mu_{0}=\mu_{\rm min}+\dfrac{\mu_{\rm max}\left(\dfrac{T}{300}\right)^\nu-\mu_{\rm
            min}}{1+\left(\dfrac{T}{300}\right)^\xi \left(\dfrac{N_{\rm total}}{N_{\rm ref}}\right)^\alpha}
\end{equation}
where $N_{\rm total}=N_A+N_D$ is the total impurity concentration.
\par
Default parameters for Si, GaAs and Ge are listed below:
\par

\begin{wtable}[-0.5cm]{lllllll}
\caption{\label{tab:Equation:Mobility:Analytic:Param}Default parameter values of the analytic mobility model}\\
\toprule
 Symbol
& Parameter
& Unit
& Si:n
& Si:p
& GaAs:n
& GaAs:p\\
\hline
 $\mu_{\rm min}$
& $\mathbf{MUN.MIN}$ / $\mathbf{MUP.MIN}$
& $cm^2V^{-1}s^{-1}$
& 55.24
& 49.70
& 0.0
& 0.0
\\
 $\mu_{\rm max}$
& $\mathbf{MUN.MAX}$ / $\mathbf{MUP.MAX}$
& $cm^2V^{-1}s^{-1}$
& 1429.23
& 479.37
& 8500.0
& 400.0
\\
 $\nu$
& $\mathbf{NUN}$ / $\mathbf{NUP}$
& -
& -2.3
& -2.2
& -1.0
& -2.1
\\
 $\xi$
& $\mathbf{XIN}$ / $\mathbf{XIP}$
& -
& -3.8
& -3.7
& 0.0
& 0.0
\\
 $\alpha$
& $\mathbf{ALPHAN}$ / $\mathbf{ALPHAP}$
& -
& 0.73
& 0.70
& 0.436
& 0.395
\\
 $N_{\rm ref}$
& $\mathbf{NREFN}$ / $\mathbf{NREFP}$
& $cm^{-3}$
& 1.072e17
& 1.606e17
& 1.69e17
& 2.75e17\\
\bottomrule
\end{wtable}

In GENIUS, the analytic model is the simplest mobility model, and is available for a wide range of materials. For some materials, such as silicon, some more advanced mobility models are available.


\par
\par
\subsubsection{Masetti Analytic Model}
\label{sec:Equation:Mobility:Bulk:Masetti}
\index{mobility!Masetii model}The doping-dependent low-field mobility model proposed by Masetti et
          al.\cite[Masetti1983]{} is an alternative to the default analytic model. The general expression
          for the low-field mobility is
\par
\begin{equation}
\mu_{\rm dop} = \mu_{\rm min1} \exp\left( -\frac{P_c} {N_{\rm tot}} \right) + \frac{\mu_{\rm
            const} - \mu_{\rm min2}}{1+\left( N_{\rm tot}/C_r \right)^\alpha } - \frac{\mu_1}{1 + \left( C_s/N_{\rm tot}
            \right)^\beta }
\end{equation}
where $N_{\rm tot}$ is the total doping concentration. The term
$\mu_{\rm const}$ is the temperature-dependent, phonon-limited mobility
\par
\begin{equation}
\mu_{\rm const} = \mu_{\rm max} \left( \frac{T}{300} \right)^{\zeta}
\end{equation}
where $T$ is the lattice temperature.
\par
The parameters of the Masetti model is listed in \ref{tab:Equation:Mobility:Masetti:Param},
p. \pageref{tab:Equation:Mobility:Masetti:Param}.
          The Masetti model is the default mobility model for the 4H-SiC material.
\par

\begin{wtable}{lllll}
\caption{\label{tab:Equation:Mobility:Masetti:Param}Parameters of the Masetti mobility model}\\
\toprule
 Symbol
& Parameter
& Unit
& 4H-SiC:n
& 4H-SiC:p\\
\hline
$\mu_{\rm max}$
& $\mathbf{MUN.MAX}$ / $\mathbf{MUP.MAX}$
& $cm^2V^{-1}s^{-1}$
& 947.0
& 124.0
\\
 $\zeta$
& $\mathbf{MUN.ZETA}$ / $\mathbf{MUP.ZETA}$
& -
& 1.962
& 1.424
\\
 $\mu_{\rm min1}$
& $\mathbf{MUN.MIN1}$ / $\mathbf{MUP.MIN1}$
& $cm^2V^{-1}s^{-1}$
& 0
& 15.9
\\
 $\mu_{\rm min2}$
& $\mathbf{MUN.MIN2}$ / $\mathbf{MUP.MIN2}$
& $cm^2V^{-1}s^{-1}$
& 0
& 15.9
\\
 $\mu_1$
& $\mathbf{MUN1}$ / $\mathbf{MUP1}$
& $cm^2V^{-1}s^{-1}$
& 0
& 0
\\
 $P_c$
& $\mathbf{PCN}$ / $\mathbf{PCP}$
& $cm^{-3}$
& 0
& 0
\\
 $C_r$
& $\mathbf{CRN}$ / $\mathbf{CRP}$
& $cm^{-3}$
& $1.94\times 10^{17}$
& $1.76\times 10^{19}$
\\
 $C_s$
& $\mathbf{CSN}$ / $\mathbf{CSP}$
& $cm^{-3}$
& 0
& 0
\\
 $\alpha$
& $\mathbf{MUN.ALPHA}$ / $\mathbf{MUP.ALPHA}$
& -
& 0.61
& 0.34
\\
 $\beta$
& $\mathbf{MUN.BETA}$ / $\mathbf{MUP.BETA}$
& -
& 0
& 0\\
 $\beta$
& $\mathbf{MUN.BETA}$ / $\mathbf{MUP.BETA}$
& -
& 0
& 0\\
 $\beta$
& $\mathbf{MUN.BETA}$ / $\mathbf{MUP.BETA}$
& -
& 0
& 0\\
 $\beta$
& $\mathbf{MUN.BETA}$ / $\mathbf{MUP.BETA}$
& -
& 0
& 0\\
 $\beta$
& $\mathbf{MUN.BETA}$ / $\mathbf{MUP.BETA}$
& -
& 0
& 0\\
 $\beta$
& $\mathbf{MUN.BETA}$ / $\mathbf{MUP.BETA}$
& -
& 0
& 0\\
 $\beta$
& $\mathbf{MUN.BETA}$ / $\mathbf{MUP.BETA}$
& -
& 0
& 0\\
 $\beta$
& $\mathbf{MUN.BETA}$ / $\mathbf{MUP.BETA}$
& -
& 0
& 0\\
 $\beta$
& $\mathbf{MUN.BETA}$ / $\mathbf{MUP.BETA}$
& -
& 0
& 0\\
 $\beta$
& $\mathbf{MUN.BETA}$ / $\mathbf{MUP.BETA}$
& -
& 0
& 0\\
\bottomrule
\end{wtable}

\subsubsection{Philips Mobility Model}
\label{sec:Equation:Mobility:Bulk:Philips}
\index{mobility!Philips model}Another low field mobility model implemented into GENIUS is the Philips Unified Mobility model
\cite[Klaassen1992-1]{},\cite[Klaassen1992-2]{}. This model takes into account the
          distinct acceptor and donor scattering, carrier-carrier scattering and carrier screening, which is recommended
          for bipolar devices simulation.
\par
The electron mobility is described by the following expressions:
\par
\begin{equation}
{\mu_{0,n}}^{-1} = {\mu_{{\rm Lattice},n}}^{-1} + {\mu _{D + A + p}}^{-1}
\end{equation}
where $\mu_{0,n}$ is the total low field electron mobilities, $\mu_{{\rm Lattice},n}$
is the electron mobilities due to lattice scattering, $\mu_{D + A + p}$
is the electron and hole mobilities due to donor (D), acceptor (A),
          screening (P) and carrier-carrier scattering.
\par
\begin{equation}
\mu_{{\rm Lattice},n}=\mu_{\rm max} \left( \frac{T}{300} \right)^{-2.285}
\end{equation}
\begin{equation}
\mu _{D + A + p} = \mu_{1,n} \left( \frac{N_{{\rm sc,}n}}{N_{{\rm sc,eff,}n}} \right) \left(
            \frac{N_{\rm ref}}{N_{{\rm sc,}n} } \right)^\alpha + \mu_{2,n} \left( \frac{n + p}{N_{{\rm sc,eff,}n}}
            \right)
\end{equation}
The parameters $\mu_{1,n}$ and $\mu_{2,n}$ are
          given as:
\par
\begin{subequations}
\begin{align}
 \mu_{1,n}  = \frac{ \mu_{\rm max }^2 }{ \mu_{\rm max } - \mu_{\rm min } }
            \left(\frac{T}{300} \right)^{3\alpha - 1.5} \\
 \mu_{2,n}  = \frac{ \mu_{\rm max } \cdot \mu_{\rm min } }{ \mu_{\rm max } - \mu_{\rm min}
            } \left( \frac{300}{T} \right)^{1.5}
\end{align}
\end{subequations}
where ${N_{{\rm sc,}n} }$ and ${N_{{\rm
          sc,eff,}n}}$ is the impurity-carrier scattering concentration and effect impurity-carrier scattering
          concentration given by:
\par
\begin{subequations}
\begin{align}
N_{{\rm sc,}n} = {N_D}^* + N_A^* + p \\
N_{{\rm sc,eff,}n} = {N_D}^* + {N_A}^* G\left( {P_n } \right) + \frac{p}{ F\left( {P_n } \right)
            }
\end{align}
\end{subequations}
where $N_D^*$ and $N_A^*$ take ultra-high doping
          effects into account and are defined by:
\par
\begin{subequations}
\begin{align}
 {N_D}^* = N_D \left( 1 + \dfrac{1}{C_D + \left( \dfrac{N_{D,{\rm ref}}}{N_D} \right)^2 }
            \right) \\
 {N_A}^* = N_A \left( 1 + \dfrac{1}{C_A + \left( \dfrac{N_{A,{\rm ref}}}{N_A } \right)^2 }
            \right)
\end{align}
\end{subequations}
The screening factor functions $G\left( P_n \right)$ and $F\left( P_n \right)$
take the repulsive potential for acceptors and the finite mass of
          scattering holes into account.
\par
\begin{widetext}
\begin{equation}
G\left( P_n \right) = 1 - \frac{0.89233}{\left[ 0.41372 + P_n \left(
            \frac{m_0}{m_e}\frac{T}{300} \right)^{0.28227} \right]^{0.19778} } + \frac{0.005978}{\left[ P_n \left(
            \frac{m_e}{m_0}\frac{T}{300} \right)^{0.72169} \right]^{1.80618}}
\end{equation}
\end{widetext}
\begin{equation}
F\left( P_n \right) = \frac{0.7643{P_n}^{0.6478} + 2.2999 + 6.5502\frac{m_e}{m_h} }
            {{P_n}^{0.6478} + 2.3670 - 0.8552\frac{m_e}{m_h} }
\end{equation}
The $P_n$ parameter that takes screening effects into account is given
          by:
\par
\begin{equation}
P_n = \left[ {\frac{ f_{cw} }{ N_{\rm sc,ref} \cdot {N_{{\rm sc,}n}}^{-2/3} } + \dfrac{ f_{BH}
            }{ \dfrac{ N_{\rm c,ref} } {n + p}\left( \dfrac{m_e}{m_0} \right)}} \right]^{-1} \left( \frac{T}{300}
            \right)^2
\end{equation}
Similar expressions hold for holes. The default parameters for Philips model are listed in
\ref{tab:Equation:Mobility:Philips:Param}, p. \pageref{tab:Equation:Mobility:Philips:Param}:
\par

\begin{wtable}{lllll}
\caption{\label{tab:Equation:Mobility:Philips:Param}Default values of Philips mobility model parameters} \\
\toprule
 Symbol
& Parameter
& Unit
& Si:n
& Si:p\\
\hline
 $\mu_{\rm min}$
& $\mathbf{MMNN.UM}$ / $\mathbf{MMNP.UM}$
& $cm^2V^{-1}s^{-1}$
& 55.24
& 49.70
\\
 $\mu_{\rm max}$
& $\mathbf{MMXN.UM}$ / $\mathbf{MMXP.UM}$
& $cm^2V^{-1}s^{-1}$
& 1417.0
& 470.5
\\
 $\alpha$
& $\mathbf{ALPN.UM}$ / $\mathbf{ALPP.UM}$
& -
& 0.68
& 0.719
\\
 $N_{\rm ref}$
& $\mathbf{NRFN.UM}$ / $\mathbf{NRFP.UM}$
& $cm^{-3}$
& 9.68e16
& 2.23e17
\\
 $C_D$
& $\mathbf{CRFD.UM}$
& -
& 0.21
& 0.21
\\
 $C_A$
& $\mathbf{CRFA.UM}$
& -
& 0.5
& 0.5
\\
 $N_{\rm D,ref}$
& $\mathbf{NRFD.UM}$
& $cm^{-3}$
& 4.0e20
& 4.0e20
\\
 $N_{\rm A,ref}$
& $\mathbf{NRFA.UM}$
& $cm^{-3}$
& 7.2e20
& 7.2e20
\\
 $m_e$
& $\mathbf{me_over_m0}$
& $m_0$
& 1.0
& -
\\
 $m_h$
& $\mathbf{mh_over_m0}$
& $m_0$
& -
& 1.258
\\
 $f_{cw}$
&
& -
& 2.459
& 2.459
\\
 $f_{BH}$
&
& -
& 3.828
& 3.828
\\
 $N_{\rm sc, ref}$
& $\mathbf{NSC.REF}$
& $cm^{-2}$
& 3.97e13
& 3.97e13
\\
 $N_{\rm c,ref}$
& $\mathbf{CAR.REF}$
& $cm^{-3}$
& 1.36e20
& 1.36e20\\
\bottomrule
\end{wtable}

In the actual code, Philips model is corrected by Caughey-Thomas expression for taking high field
          velocity saturation effects into account. This model can be loaded by
$\mathbf{Philips}$ keyword
          in the $\mathbf{PMI}$ statements.
\par
\subsubsection{Velocity Saturation}
\label{sec:Equation:Mobility:Bulk:VSat}
\index{velocity saturation}
\par
\marginhead{Silicon-like materials}\index{velocity saturation!Caughey-Thomas model}For silicon-like materials, the Caughey-Thomas expression
\cite[Caughey1967]{}, is
            used:
\par
\par
\begin{equation}
\mu = \dfrac{\mu _{0} }{\left[ 1 + \left( \dfrac{\mu _{0} E_{\parallel} }{v_{\rm sat} }
            \right)^\beta \right]^{1/\beta} }
\end{equation}
where $E_{\parallel}$ is the electric field parallel to current flow.
$v_{\rm sat}$ is the saturation velocities for electrons or holes. They are
          computed by default from the expression:
\par
\begin{equation}
v_{\rm sat} (T) = \frac{ v_{\rm sat0} }{ 1 + \alpha \cdot \exp \left( \frac{T}{600} \right)
            }
\end{equation}
The parameters and the default values for silicon is listed in \ref{tab:Equation:Vsat:Si:Param},
p. \pageref{tab:Equation:Vsat:Si:Param}.

\begin{wtable}{lllll}
\caption{\label{tab:Equation:Vsat:Si:Param}Velocity saturation parameters of silicon-like materials} \\
\toprule
 Symbol
& Parameter
& Unit
& Si:n
& Si:p\\
\hline
$v_{\rm sat0}$
& $\mathbf{VSATN0}$ / $\mathbf{VSATP0}$
& \si{\centi\meter\per\second}
& $2.4\times10^7$
& $2.4\times10^7$
\\
 $\beta$
& $\mathbf{BETAN}$ / $\mathbf{BETAP}$
& -
& 2.0
& 1.0
\\
 $\alpha$
& $\mathbf{VSATN.A}$ / $\mathbf{VSATP.A}$
& -
& 0.8
& 0.8\\
\bottomrule
\end{wtable}

\marginhead{GaAs-like materials}\index{velocity saturation!GaAs-like}For GaAs-like materials, another expression due to
\cite[Barnes1976]{} is used to
            describe the negative differential resistance:
\par
\par
\begin{equation}
u = \frac{ \mu _{0} + \dfrac{ v_{sat} }{ E_{\parallel} } \left( \dfrac{E_{\parallel} }{ E_{0} }
            \right)^4 } {1 + \left( \dfrac{E_{\parallel} }{ E_{0} } \right)^4 }
\end{equation}
where $E_{0}$ is the reference field, and the saturation velocity
\par
\begin{equation}
v_{sat} (T) = A_{\rm vsat} - B_{\rm vsat} T
\end{equation}
The negative differential property of carrier mobility is described in this model. When electric field
          increases in this model, the carrier drift velocity ($\mu E_{\parallel}$) reaches
          a peak and then begins to decrease at high fields due to the transferred electron effect.
\par
The parameters are listed in \ref{tab:Equation:Vsat:GaAs:Param},
p. \pageref{tab:Equation:Vsat:GaAs:Param}.

\begin{wtable}{lllll}
\caption{\label{tab:Equation:Vsat:GaAs:Param}Velocity saturation parameters of GaAs-like materials} \\
\toprule
 Symbol
& Parameter
& Unit
& GaAs:n
& GaAs:p\\
\hline
 $A_{\rm vsat}$
& $\mathbf{VSATN.A}$ / $\mathbf{VSATP.B}$
& \si{\centi\meter\per\second}
& $1.13\times10^{7}$
& $1.13\times10^{7}$
\\
 $B_{\rm vsat}$
& $\mathbf{VSATN.A}$ / $\mathbf{VSATP.B}$
& \si{\centi\meter\per\second\per\kelvin}
& $1.2\times10^{4}$
& $1.2\times10^{4}$\\
\bottomrule
\end{wtable}

\par
\marginhead{GaAs-specific model}\index{velocity saturation!hyper-tangent model}When using this model for GaAs MESFET device simulation, the negative differential property
            may cause the drain output characteristics (current vs. voltage) exhibit an unrealistic oscillation
            behavior. Another model to describe high field effects developed by Yeager
\cite[Yeager1986]{} can be used.
\par
\par
\begin{equation}
\mu = \frac{v_{\rm sat} }{ E_{\parallel}} \tanh \left( \frac{\mu _0 E_{\parallel} } {v_{\rm sat}
            } \right)
\end{equation}
This GaAs-specific model can be loaded by $\mathbf{Hypertang}$ keyword in
$\mathbf{PMI}$ statement.
\par
\marginhead{4H-SiC-specific model}For 4H-SiC, the saturation velocity is calculated with the following formula
\par
\par
\begin{equation}
v_{sat} (T) = A_{\rm vsat} - B_{\rm vsat} \left( \frac{T}{300} \right)
\end{equation}
where the parameters are listed in
\par
\begin{wtable}{llllll}
\caption{\label{tab:Equation:Vsat:4HSiC:Param}Velocity saturation parameters of 4H-SiC} \\
\toprule
 Symbol
& Parameter
& Unit
& 4H-SiC:n
& 4H-SiC:p\\
\hline
 $A_{\rm vsat}$
& $\mathbf{VSATN.A}$ / $\mathbf{VSATP.B}$
& \si{\centi\meter\per\second}
& $1.07\times10^{7}$
& $8.37\times10^{6}$
\\
 $B_{\rm vsat}$
& $\mathbf{VSATN.A}$ / $\mathbf{VSATP.B}$
& \si{\centi\meter\per\second}
& $0$
& $0$\\
\bottomrule
\end{wtable}

\subsection{Unified Mobility Models}
The other family of mobility models are the unified mobility models. The effect of high transverse and
        parallel E-field is an integral part in the design of these mobility models. As a result, these models are
        recommended for silicon MOSFET simulation. On the other hand, the availability of the unified models is limited
        to a few materials, such as silicon and silicon-germanium.
\par
\subsubsection{Lombardi Surface Mobility Model}
\label{sec:Equation:Mobility:Unified:Lombardi}
\index{mobility!Lombardi model}Along an insulator-semiconductor interface, the carrier mobilities can be substantially lower
          than in the bulk of the semiconductor due to surface-related scattering. If no surface degradation is
          considered, the drain-source current may exceed about
30\% for MOS
          simulation.
\par
The Lombardi mobility model \cite[Lombardi1988]{} is an empirical model that is able to
          describe the carrier mobility in the MOSFET inversion layer. The Lombardi model consists of three
          components
\par
\begin{itemize}
\item $\mu_b$, the doping-dependent bulk mobility. This component mainly
                accounts for the ionized impurity scattering.
\par
\item $\mu_{\rm ac}$, the mobility degradation due to acoustic phonon
                scattering in the inversion layer. Due to the quantum confinement in the potential well at the
                interface, this mobility degradation is a strong function of the transverse electric field.
\par
\item $\mu_{\rm sr}$, the mobility degradation due to the surface roughness
                scattering. This component is also a strong function of the transverse electric field.
\par
\end{itemize}
To obtain the final value of carrier mobility, the three components are combined using the
          Matthiessen's rule:
\par
\begin{equation}
{\mu_{\rm s}}^{ - 1} = {\mu_b}^{-1} + {\mu _{\rm ac}}^{ - 1} + {\mu _{\rm
            sr}}^{-1}.
\end{equation}
\marginhead{Bulk Component}The bulk mobility component in Lombardi's model is similar to that of Masetti's model, which
            reads
\par
\par
\begin{equation}
\mu_b = \mu_0 \exp\left( -\frac{P_c} {N_{\rm tot}} \right) + \frac{\mu_{\rm max} -
            \mu_0}{1+\left( N_{\rm tot}/C_r \right)^\alpha } - \frac{\mu_1}{1 + \left( C_s/N_{\rm tot} \right)^\beta
            }
\end{equation}
\begin{equation}
\mu_{\rm max} = \mu_2 \left( \frac{T}{300} \right)^\zeta
\end{equation}
\marginhead{Acoustic Phonon Component}The acoustic phonon limited mobility component is
\par
\par
\begin{equation}
\mu_{\rm ac} = \frac{B}{E_\bot} + \dfrac{C \cdot {N_{\rm total}}^{\lambda} }{ T \sqrt[3]{E_\bot}
            }
\end{equation}
where $E_\bot$ is the transverse component of the electric field.
\par
\marginhead{Surface Roughness Component}Finally, the surface roughness limited mobility is expressed as
\par
\par
\begin{equation}
\label{eq:Equation:Mobility:Lombardi:SR}
\mu_{\rm sr} = \frac{D}{E_{\bot}^{\gamma} }.
\end{equation}
The Lombardi model is uses the Caughey-Thomas model for velocity saturation calculation, see
\nameref{sec:Equation:Mobility:Bulk:VSat}, p. \pageref{sec:Equation:Mobility:Bulk:VSat}
for details.
\par
The parameters used in the Lombardi model is summarized in \ref{tab:Equation:Mobility:Lombardi:Param},
p. \pageref{tab:Equation:Mobility:Lombardi:Param}.
\par

\begin{wtable}{llllll}
\caption{\label{tab:Equation:Mobility:Lombardi:Param}Parameters of Lombardi mobility model}\\
\toprule
 Symbol
& Parameter
& Unit
& Si:n
& Si:p\\
\hline
 $\alpha$
& $\mathbf{EXN1.LSM}$ / $\mathbf{EXP1.LSM}$
& -
& 0.680
& 0.719
\\
 $\beta$
& $\mathbf{EXN2.LSM}$ / $\mathbf{EXP2.LSM}$
& -
& 2.0
& 2.0
\\
 $\zeta$
& $\mathbf{EXN3.LSM}$ / $\mathbf{EXP3.LSM}$
& -
& 2.5
& 2.2
\\
 $\lambda$
& $\mathbf{EXN4.LSM}$ / $\mathbf{EXP4.LSM}$
& -
& 0.125
& 0.0317
\\
 $\gamma$
& $\mathbf{EXN8.LSM}$ / $\mathbf{EXP8.LSM}$
& -
& 2.0
& 2.0
\\
 $\mu_0$
& $\mathbf{MUN0.LSM}$ / $\mathbf{MUP0.LSM}$
& $cm^2V^{-1}s^{-1}$
& $52.2$
& $44.9$
\\
 $\mu_1$
& $\mathbf{MUN1.LSM}$ / $\mathbf{MUP1.LSM}$
& $cm^2V^{-1}s^{-1}$
& $43.4$
& $29.0$
\\
 $\mu_2$
& $\mathbf{MUN2.LSM}$ / $\mathbf{MUP2.LSM}$
& $cm^2V^{-1}s^{-1}$
& $1417.0$
& $470.5$
\\
 $P_c$
& $\mathbf{PC.LSM}$
& $cm^{-3}$
& 0 (fixed)
& $9.23\times 10^{16}$
\\
 $C_r$
& $\mathbf{CRN.LSM}$ / $\mathbf{CRP.LSM}$
& $cm^{-3}$
& $9.68\times 10^{16}$
& $2.23\times 10^{17}$
\\
 $C_s$
& $\mathbf{CSN.LSM}$ / $\mathbf{CSP.LSM}$
& $cm^{-3}$
& $3.43\times 10^{20}$
& $6.10\times 10^{20}$
\\
 $B$
& $\mathbf{BN.LSM}$ / $\mathbf{BP.LSM}$
& \si{\centi\meter\per\second}
& $4.75\times 10^{7}$
& $9.93\times 10^{6}$
\\
 $C$
& $\mathbf{CN.LSM}$ / $\mathbf{CP.LSM}$
& -
& $1.74\times 10^{5}$
& $8.84\times 10^{5}$
\\
 $D$
& $\mathbf{DN.LSM}$ / $\mathbf{DP.LSM}$
& -
& $5.82\times 10^{14}$
& $2.05\times 10^{14}$
\\
 $v_{\rm sat0}$
& $\mathbf{VSATN0}$ / $\mathbf{VSATP0}$
& \si{\centi\meter\per\second}
& $2.4\times 10^7$
& $2.4\times 10^7$
\\
 $\beta$
& $\mathbf{BETAN}$ / $\mathbf{BETAP}$
& -
& 2.0
& 1.0
\\
 $\alpha$
& $\mathbf{VSATN.A}$ / $\mathbf{VSATP.A}$
& -
& 0.8
& 0.8\\
\bottomrule
\end{wtable}

Hewlett-Packard mobility model can be loaded by $\mathbf{Lombardi}$
keyword in the $\mathbf{PMI}$ statement.
\par
\subsubsection{Lucent High Field Mobility Model}
\label{sec:Equation:Mobility:Unified:Lucent}
\index{mobility!Lucent model}The Lucent Mobility model \cite[Darwish1997]{}
is an all-inclusive model which
          suitable for MOS simulation. This model incorporates Philips Unified Mobility model and the Lombardi Surface
          Mobility model, as well as accounting for high field effects. For low longitudinal field, the carrier mobility
          is given by Matthiessen's rule:
\par
\begin{equation}
\mu_{0} = \left[ \frac{1}{\mu_{\rm b}} + \frac{1}{\mu_{\rm ac} } + \frac{1}{\mu_{\rm sr} }
            \right]^{-1}
\end{equation}
where $\mu_b$ is the bulk mobility comes from the Philips model, and
$\mu_{\rm ac}$ and $\mu_{\rm sr}$ come from the Lombardi
          model. The details of these models are described in \nameref{sec:Equation:Mobility:Bulk:Philips},
p. \pageref{sec:Equation:Mobility:Bulk:Philips} and \nameref{sec:Equation:Mobility:Unified:Lombardi},
p. \pageref{sec:Equation:Mobility:Unified:Lombardi}, and will not be repeated here.
\par
There is however a modification to the surface roughness formula
\eqref{eq:Equation:Mobility:Lombardi:SR}, p. \pageref{eq:Equation:Mobility:Lombardi:SR}. The constant exponent
$\lambda$ is
          replaced by the following function
\par
\begin{equation}
\lambda = A + \dfrac{ F\cdot (n+p) } { N_{\rm tot} ^ \nu }.
\end{equation}
The Lucent model uses the Caughey-Thomas model for velocity saturation calculation, see
\nameref{sec:Equation:Mobility:Bulk:VSat}, p. \pageref{sec:Equation:Mobility:Bulk:VSat}
for details.
\par
The parameters of the Lucent model is listed in \ref{tab:Equation:Mobility:Lucent:Param},
p. \pageref{tab:Equation:Mobility:Lucent:Param}.
\par

\begin{wtable}{lllll}
\caption{\label{tab:Equation:Mobility:Lucent:Param}Default values of Lucent mobility model parameters} \\
\toprule
 Symbol
& Parameter
& Unit
& Si:n
& Si:p\\
\hline
 $\mu_{\rm min}$
& $\mathbf{MMNN.UM}$ / $\mathbf{MMNP.UM}$
& $cm^2V^{-1}s^{-1}$
& 55.24
& 49.70
\\
 $\mu_{\rm max}$
& $\mathbf{MMXN.UM}$ / $\mathbf{MMXP.UM}$
& $cm^2V^{-1}s^{-1}$
& 1417.0
& 470.5
\\
 $\alpha$
& $\mathbf{ALPN.UM}$ / $\mathbf{ALPP.UM}$
& -
& 0.68
& 0.719
\\
 $N_{\rm ref}$
& $\mathbf{NRFN.UM}$ / $\mathbf{NRFP.UM}$
& $cm^{-3}$
& 9.68e16
& 2.23e17
\\
 $C_D$
& $\mathbf{CRFD.UM}$
& -
& 0.21
& 0.21
\\
 $C_A$
& $\mathbf{CRFA.UM}$
& -
& 0.5
& 0.5
\\
 $N_{\rm D,ref}$
& $\mathbf{NRFD.UM}$
& $cm^{-3}$
& 4.0e20
& 4.0e20
\\
 $N_{\rm A,ref}$
& $\mathbf{NRFA.UM}$
& $cm^{-3}$
& 7.2e20
& 7.2e20
\\
 $m_e$
& $\mathbf{me_over_m0}$
& $m_0$
& 1.0
& -
\\
 $m_h$
& $\mathbf{mh_over_m0}$
& $m_0$
& -
& 1.258
\\
 $f_{cw}$
&
& -
& 2.459
& 2.459
\\
 $f_{BH}$
&
& -
& 3.828
& 3.828
\\
 $N_{\rm sc, ref}$
& $\mathbf{NSC.REF}$
& $cm^{-2}$
& 3.97e13
& 3.97e13
\\
 $N_{\rm c,ref}$
& $\mathbf{CAR.REF}$
& $cm^{-3}$
& 1.36e20
& 1.36e20
\\
 $v_{\rm sat0}$
& $\mathbf{VSATN0}$ / $\mathbf{VSATP0}$
& \si{\centi\meter\per\second}
& $2.4\times10^7$
& 2.47
\\
 $\lambda$
& $\mathbf{EXN4.LUC}$ / $\mathbf{EXP4.LUC}$
& -
& 0.0233
& 0.0119
\\
 $\nu$
& $\mathbf{EXN9.LUC}$ / $\mathbf{EXP9.LUC}$
& -
& 0.0767
& 0.123
\\
 $A$
& $\mathbf{AN.LUC}$ / $\mathbf{AP.LUC}$
& -
& $2.58$
& $2.18$
\\
 $B$
& $\mathbf{BN.LUC}$ / $\mathbf{BP.LUC}$
& \si{\centi\meter\per\second}
& $3.61\times 10^{7}$
& $1.51\times 10^{7}$
\\
 $C$
& $\mathbf{CN.LUC}$ / $\mathbf{CP.LUC}$
& -
& $1.70\times 10^{4}$
& $4.18\times 10^{3}$
\\
 $D$
& $\mathbf{DN.LUC}$ / $\mathbf{DP.LUC}$
& -
& $3.58\times 10^{18}$
& $3.58\times 10^{18}$
\\
 $F$
& $\mathbf{FN.LUC}$ / $\mathbf{FP.LUC}$
& -
& $6.85\times 10^{-21}$
& $7.82\times 10^{-21}$
\\
 $K$
& $\mathbf{KN.LUC}$ / $\mathbf{KP.LUC}$
& -
& $1.70$
& $0.90$
\\
 $v_{\rm sat0}$
& $\mathbf{VSATN0}$ / $\mathbf{VSATP0}$
& \si{\centi\meter\per\second}
& $2.4\times10^7$
& $2.4\times10^7$
\\
 $\beta$
& $\mathbf{BETAN}$ / $\mathbf{BETAP}$
& -
& 2.0
& 1.0
\\
 $\alpha$
& $\mathbf{VSATN.A}$ / $\mathbf{VSATP.A}$
& -
& 0.8
& 0.8\\
\bottomrule
\end{wtable}

The Lucent model can be loaded by $\mathbf{Lucent}$ keyword in the
$\mathbf{PMI}$ statements. This is an accurate model recommended for MOS devices. However, its computation cost is higher
          than other mobility models. At the same time, it is also less numerically stable.
\par
\par
\subsubsection{Hewlett-Packard High Field Mobility Model}
\label{sec:Equation:Mobility:Unified:HP}
\index{mobility!HP model}It is reported that Hewlett-Packard mobility model
\cite[Cham1986]{} achieves the
          same accuracy as Lucent model with relatively small computational burden in the MOS simulation.
\par
This model also takes into account dependence on electric fields both parallel ($E_\parallel$) and perpendicular ($E_\bot$) to the direction
          of current flow. The mobility is calculated from
\par
\begin{equation}
\mu = \dfrac{\mu _{\bot} } {\sqrt {1 + \dfrac{ \left( \dfrac{\mu_{\bot} E_{\parallel} }{v_{c} }
            \right)^2 } {\dfrac{\mu_{\bot} E_{\parallel} }{v_{c} } + \gamma } + \dfrac{\mu_{\bot} E_{\parallel} } {v_{s}
            } } }
\end{equation}
where the tranverse field dependent component $\mu_{\bot}$ is given
          as:
\par


\begin{equation}
\mu_{\bot} = \  \mu_{0}   \text{if } N_{\rm total} > N_{\rm ref} \\
             \dfrac{\mu_0}{ 1+\dfrac{E_{\bot}}{E_{\rm ref}} }
\end{equation}


The low field mobility $\mu_{0}$ is calculated from the $\mathbf{Analytic}$
model, as described in \nameref{sec:Equation:Mobility:Bulk:Analytic},
p. \pageref{sec:Equation:Mobility:Bulk:Analytic}.
\par
The parameters of the Hewlett-Packard mobility model and its default values for silicon is listed in
\ref{tab:Equation:Mobility:HP:Param}, p. \pageref{tab:Equation:Mobility:HP:Param}.
\par

\begin{wtable}{lllll}
\caption{\label{tab:Equation:Mobility:HP:Param}Default values of HP mobility model parameters}\\
\toprule
 Symbol
& Parameter
& Unit
& Si:n
& Si:p\\
\hline
 $\mu_{\rm min}$
& $\mathbf{MUN.MIN}$ / $\mathbf{MUP.MIN}$
& $cm^2V^{-1}s^{-1}$
& 55.24
& 49.70
\\
 $\mu_{\rm max}$
& $\mathbf{MUN.MAX}$ / $\mathbf{MUP.MAX}$
& $cm^2V^{-1}s^{-1}$
& 1429.23
& 479.37
\\
 $\nu$
& $\mathbf{NUN}$ / $\mathbf{NUP}$
& -
& -2.3
& -2.2
\\
 $\xi$
& $\mathbf{XIN}$ / $\mathbf{XIP}$
& -
& -3.8
& -3.7
\\
 $\alpha$
& $\mathbf{ALPHAN}$ / $\mathbf{ALPHAP}$
& -
& 0.73
& 0.70
\\
 $N_{\rm ref}$
& $\mathbf{NREFN}$ / $\mathbf{NREFP}$
& $cm^{-3}$
& $1.072\times 10^{17}$
& $1.606\times 10^{17}$
\\
 $\mu_0$
& $\mathbf{MUN0.HP}$ / $\mathbf{MUP0.HP}$
& $cm^2V^{-1}s^{-1}$
& 774.0
& 250
\\
 $v_c$
& $\mathbf{VCN.NP}$ / $\mathbf{VCP.HP}$
& $cm.s^{-1}$
& $4.9\times 10^6$
& $2.928 \times 10^6$
\\
 $v_s$
& $\mathbf{VSN.NP}$ / $\mathbf{VSP.HP}$
& $cm.s^{-1}$
& $1.036\times 10^7$
& $1.2\times 10^7$
\\
 $\gamma$
& $\mathbf{GN.HP}$ / $\mathbf{GP.HP}$
& -
& 8.8
& 1.6
\\
 $N_{\rm ref}$
& $\mathbf{NREFN}$ / $\mathbf{NREFP}$
& $cm^{-3}$
& $5\times 10^{17}$
& $5\times 10^{17}$
\\
 $E_{\rm ref}$
& $\mathbf{ECN.HP}$ / $\mathbf{ECP.HP}$
& $V.cm^{-1}$
& $5.5\times 10^5$
& $2.78\times 10^5$\\
\bottomrule
\end{wtable}

The Hewlett-Packard mobility model can be loaded by $\mathbf{HP}$
keyword in the $\mathbf{PMI}$ statement.
\par
\subsection{Mobility Models of Complex Compound Semiconductors}
[[TODO]]
\par
\subsection{Carrier Temperature Based Mobility Model}
We should notice here, all the above mobility models are developed under the framework of DD method. Since
        DD is an approximate model for semiconductor, the difference between DD model and real device is corrected by
        mobility models! These mobility model contains some physical model that DD does not consider. For example, the
        high field correction has already contains the effect of hot carriers. The surface mobility for MOSFET not only
        considers the mobility degrade due to surface roughness, but also contains the effect caused by carrier
        concentration decrease due to quantum well in inverse layer. These corrections extended the application range of
        DD model, also make the mobility model rather complex.
\par
When the physical model is more accurate, the carrier mobility model can be less complicated. Thus, the
        mobility models suitable for DD model may NOT be suitable for energy balance model. There are some mobility
        models developed special for energy balance model \cite[PISCES-2ET]{}. However, they are yet to be
        implemented in GENIUS.
\par
\section{Generation Model}
\subsection{Impact Ionization}
\label{sec:Equation:Generation:II}
\index{Impact ionization}The generation rate of electron-hole pairs due to the carrier impact ionization (II) is generally
        modeled as \cite[Sze1981]{}:
\par
\begin{equation}
G^{II} = \alpha_n \dfrac{\left\vert \vec{J}_n \right\vert }{q} + \alpha_p \dfrac{ \left\vert
          \vec{J}_p \right\vert }{q}
\end{equation}
where $\alpha_n$ and $\alpha_p$ are electron and
        hole ionization coefficients, related with electrical field, material and temperature.
\par
Three models are implemented in Genius to calculate the ionization coefficient.
\par
\subsubsection{Selberherr Model}
\index{Impact ionization!Selberherr model}Selberherr gives an empirical formula
\cite[Selberherr1984]{}, based on the
          expression derived by Chynoweth\cite[Chynoweth1958]{}:
\par
\begin{equation}
\alpha _{n,p} = \alpha _{n,p}^\infty (T)\exp \left[ - \left( \frac{E_{n,p}^{\rm Crit} }{E_{n,p}
            } \right)^{\gamma_{n,p}} \right]
\end{equation}
where $E_{n,p}$ is the magnitude of driving fields. When $\mathbf{EdotJ}$
model is used, $E_{n,p}$ can be given by:
\par
\begin{subequations}
\begin{align}
 E_n = \frac{ \vec{E} \cdot \vec{J}_n } { \left\vert \vec{J}_n \right\vert }\\
 E_p = \frac{ \vec{E} \cdot \vec{J}_p } { \left\vert \vec{J}_p \right\vert }
\end{align}
\end{subequations}
and for $\mathbf{GradQf}$ model:
\par
\begin{subequations}
\begin{align}
 E_n = \left\vert \nabla \phi_{F_n} \right\vert \\
 E_p = \left\vert \nabla \phi_{F_p} \right\vert
\end{align}
\end{subequations}
where $E_{n,p}^{\rm Crit}=\dfrac{E_g}{q\lambda_{n,p}}$, for which
$\lambda_{n,p}$ are the optical-phonon mean free paths for electrons and holes given
          by:
\par
\begin{subequations}
\begin{align}
 \lambda_n (T) = \lambda_{n,0} \cdot \tanh \left( \dfrac{E_{\rm op}}{2k_b T} \right) \\
 \lambda_p (T) = \lambda_{p,0} \cdot \tanh \left( \dfrac{E_{\rm op}}{2k_b T} \right)
\end{align}
\end{subequations}
in the above expressions, $E_{op}$ is the optical-phonon energy.
$\lambda_{n,0}$ and $\lambda_{p,0}$ are the phonon mean free
          paths for electrons and holes at $300\si{\kelvin}$.
\par
The temperature dependent factors $\alpha_n^\infty$ and $\alpha_p^\infty$
are expressed as:
\par
\begin{subequations}
\begin{align}
 \alpha_n^\infty = \alpha_{n,0} + \alpha_{n,1} \cdot T + \alpha_{n,2} \cdot T^2\\
 \alpha_p^\infty = \alpha_{p,0} + \alpha_{p,1} \cdot T + \alpha_{p,2} \cdot T^2
\end{align}
\end{subequations}
The Selberherr model is the default avalanche model for many materials in Genius. One can also
          explicitly load it with the option $\mathbf{Selberherr}$
in the $\mathbf{PMI}$ statements.The parameters used for Selberherr model are listed in
\ref{tab:Equation:II:Selberherr:Param}, p. \pageref{tab:Equation:II:Selberherr:Param}.
\par

\begin{wtable}{llllll}
\caption{\label{tab:Equation:II:Selberherr:Param}Default values of Impact Ionization model parameters} \\
\toprule
 Symbol
& Parameter
& Unit
& Silicon
& GaAs
& Ge\\
\hline
 $\lambda_{n,0}$
& $\mathbf{LAN300}$
& \si{\centi\meter}
& 1.04542e-6
& 3.52724e-6
& 6.88825e-7
\\
 $\lambda_{p,0}$
& $\mathbf{LAP300}$
& \si{\centi\meter}
& 6.32079e-7
& 3.67649e-6
& 8.39505e-7
\\
 $\gamma_{n}$
& $\mathbf{EXN.II}$
& -
& 1.0
& 1.6
& 1.0
\\
 $\gamma_{p}$
& $\mathbf{EXP.II}$
& -
& 1.0
& 1.75
& 1.0
\\
 $E_{\rm op}$
& $\mathbf{OP.PH.EN}$
& \si{\eV}
& 6.3e-2
& 3.5e-2
& 3.7e-2
\\
 $\alpha_{n,0}$
& $\mathbf{N.IONIZA}$
& \si{\per\centi\meter}
& 7.030e5
& 2.994e5
& 1.55e7
\\
 $\alpha_{n,1}$
& $\mathbf{N.ION.1}$
& \si{\per\centi\meter}
& 0.0
& 0.0
& 0.0
\\
 $\alpha_{n,2}$
& $\mathbf{N.ION.2}$
& \si{\per\centi\meter}
& 0.0
& 0.0
& 0.0
\\
 $\alpha_{p,0}$
& $\mathbf{P.IONIZA}$
& \si{\per\centi\meter}
& 1.528e6
& 2.215e5
& 1e7
\\
 $\alpha_{p,1}$
& $\mathbf{P.ION.1}$
& \si{\per\centi\meter}
& 0.0
& 0.0
& 0.0
\\
 $\alpha_{p,2}$
& $\mathbf{P.ION.2}$
& \si{\per\centi\meter}
& 0.0
& 0.0
& 0.0\\
\bottomrule
\end{wtable}

\par
\subsubsection{van Overstraeten - de Man model}
\index{Impact ionization!van Overstraeten - de Man model}The model also uses the ionization coefficient derived by Chynoweth
\cite[Chynoweth1958]{}
\par
\begin{equation}
\alpha = \gamma a \exp \left( -\frac{ \gamma b }{ E } \right),
\end{equation}
where
\par
\begin{equation}
\gamma = \dfrac{\tanh \left(\dfrac{\hbar\omega_{\rm op} }{2kT_0} \right) } { \tanh \left(
            \dfrac{\hbar\omega_{\rm op}}{2kT} \right) }.
\end{equation}
The van Overstraeten - de Man model uses two sets of values for parameter
$a$ and $b$ at high and low electric field. The threshold for
          the switch is $E_0$.
\par
The parameters are listed in \ref{tab:Equation:II:Overstraeten:Param},
p. \pageref{tab:Equation:II:Overstraeten:Param}.
\par
The van Overstraeten - de Man model is can be selected with the
$\mathbf{vanOverstraetendeMan}$ option in the $\mathbf{PMI}$ command. It is the default
          model for 4H-SiC, InN, InAs and InSb materials.
\par
\begin{wtable}{lllll}
\caption{\label{tab:Equation:II:Overstraeten:Param}van Overstraeten - de Man Impact Ionization model parameters}\\
\toprule
 Parameter
& Parameter
& Unit
& 4H-SiC:n
& 4H-SiC:p\\
\hline
 $b$
& $\mathbf{AN.II.LO}$ / $\mathbf{AP.II.LO}$
& \si{\per\centi\meter}
& $4.2\times10^5$
& $4.2\times10^5$
\\
 $b$
& $\mathbf{AN.II.HI}$ / $\mathbf{AP.II.HI}$
& \si{\per\centi\meter}
& $4.2\times10^5$
& $4.2\times10^5$
\\
 $b$
& $\mathbf{BN.II.LO}$ / $\mathbf{BP.II.LO}$
& \si{\volt\per\centi\meter}
& $1.67\times10^7$
& $1.67\times10^7$
\\
 $b$
& $\mathbf{BN.II.HI}$ / $\mathbf{BP.II.HI}$
& \si{\volt\per\centi\meter}
& $1.67\times10^7$
& $1.67\times10^7$
\\
 $E_0$
& $\mathbf{E0N.II}$ / $\mathbf{EOP.II}$
& \si{\volt\per\centi\meter}
& $4\times10^5$
& $4\times10^5$
\\
 $\hbar\omega_{\rm op}$
& $\mathbf{EN.OP.PH}$ / $\mathbf{EP.OP.PH}$
& \si{\eV}
& 1.0
& 1.0\\
\bottomrule
\end{wtable}

\subsubsection{Valdinoci Model}
\index{Impact ionization!Valdinoci model}GENIUS has another Valdinoci model for silicon device which has been reported to produce correct
          temperature dependence of breakdown voltage of junction diodes as high as
$400^{\circ}C$ \cite[Valdinoci1999]{}. It can be loaded with the
$\mathbf{Valdinoci}$ option in the $\mathbf{PMI}$ statements.
\par
The electron impact ionization rate for Valdinoci model reads:
\par
\begin{equation}
\alpha _n = \frac{E_{\parallel} } { a_n (T) + b_n (T) \exp \left( \dfrac{d_n(T)}{ E_{\parallel}
            + c_n(T)} \right) }
\end{equation}
where
\par
\begin{subequations}
\begin{align}
 a_n(T) = {\bf A0N} + {\bf A1N} \cdot T^{\bf A2N} \\
 b_n(T) = {\bf B0N} \cdot \exp \left( {\bf B1N} \cdot T \right) \\
 c_n (T) = {\bf C0N} + {\bf C1N} \cdot T^{\bf C2N} + {\bf C3N} \cdot T^2 \\
 d_n (T) = {\bf D0N} + {\bf D1N} \cdot T + {\bf D2N} \cdot T^2
\end{align}
\end{subequations}
Similar expressions hold for holes. The parameters for Valdinoci model are listed in
\ref{tab:Equation:II:Valdinoci:Param}, p. \pageref{tab:Equation:II:Valdinoci:Param}.
\par
\begin{wtable}{lllll}
\caption{\label{tab:Equation:II:Valdinoci:Param}Default values of Valdinoci Impact Ionization model parameters} \\
\toprule
 Parameter
& Silicon:n
& Parameter
& Silicon:p
& Unit\\
\hline
$\mathbf{A0N}$
& 4.3383
& $\mathbf{A0P}$
& 2.376
& \si{\volt}
\\
 $\mathbf{A1N}$
& $-2.42\times10^{-12}$
& $\mathbf{A1P}$
& 0.01033
& $\si{\volt}\cdot\si{\kelvin}^{\mathbf{A2X}}$
\\
 $\mathbf{A2N}$
& 4.1233
& $\mathbf{A2P}$
& 1.0
& -
\\
 $\mathbf{B0N}$
& 0.235
& $\mathbf{B0P}$
& 0.17714
& \si{\volt}
\\
 $\mathbf{B1N}$
& 0.0
& $\mathbf{B1P}$
& $-2.178\times10^{-3}$
& \si{\per\kelvin}
\\
 $\mathbf{C0N}$
& $1.6831\times10^4$
& $\mathbf{C0P}$
& 0.0
& \si{\volt\per\centi\meter}
\\
 $\mathbf{C1N}$
& 4.3796
& $\mathbf{C1P}$
& $9.47\times10^{-3}$
& $\si{\volt\per\centi\meter}\cdot\si{\kelvin}^{-{\bf C2X}}$
\\
 $\mathbf{C2N}$
& 1.0
& $\mathbf{C2P}$
& 2.4924
& -
\\
 $\mathbf{C3N}$
& 0.13005
& $\mathbf{C3P}$
& 0.0
& \si{\volt\per\centi\meter\per\square\kelvin}
\\
 $\mathbf{D0N}$
& $1.233735\times10^6$
& $\mathbf{D0P}$
& $1.4043\times10^6$
& \si{\volt\per\centi\meter}
\\
 $\mathbf{D1N}$
& $1.2039\times10^3$
& $\mathbf{D1P}$
& $2.9744\times10^3$
& \si{\volt\per\centi\meter\per\kelvin}
\\
 $\mathbf{D2N}$
& 0.56703
& $\mathbf{D2P}$
& 1.4829
& \si{\volt\per\centi\meter\per\square\kelvin}
\\
\bottomrule
\end{wtable}

\par
\subsection{Band-to-band Tunneling}
\index{Band-to-band tunneling!Kane's model}
\par
\marginhead{Kane's model}The carrier generation by band-band tunneling
$G^{BB}$ is also considered by
          Genius, which can be expressed as \cite[Kane1959]{}\cite[Liou1990]{}:
\par
\par
\begin{equation}
G^{BB} = {\rm A.BTBT} \cdot \frac{E^2}{ \sqrt {E_g} } \cdot \exp \left( -{\rm B.BTBT} \cdot
          \dfrac{ {E_g}^{3/2} }{E} \right)
\end{equation}
where $E$ is the magnitude of electrical field.
\par
\section{High Energy Particles}
As a heavy ion passes through the device, it will interact with some silicon atoms and transfer energy to
      the semiconductor lattice, which generates electron-hole pairs along it trajectory. The simulation of high-energy
      particle and the energy deposition in semiconductor can be simulated with the Geant4 or other Monte Carlo
      simulation tool. On the other hand, the generation of electron-hole pairs and the effects to the device behavior
      must be simulated in a 3D device simulator.
\par
Assuming the proton hit the diode at $t=0\si{\second}$, and the electron-hole
      generation rate follows a Gaussian time dependence, with a maximum at
$t_{\rm max}$,
      and a characteristic time $\tau$, the carrier generation rate can be calculated
      by
\par
\begin{equation}
G = G_0 \exp \left[ -\frac{(t-t_{\rm max})^2}{2\tau^2} \right] .
\end{equation}
On the other hand, the Monte Carlo simulators such as Geant4 provides the total energy deposition data,
      which relate to $G$ by
\par
\begin{equation}
\label{eq:Particle:E}
E = \eta \int\limits_0^{\infty} G_0 \exp \left[ -\frac{(t-t_{\rm max})^2}{2\tau^2} \right]
        dt.
\end{equation}
where $\eta$ is the average energy loss for each generated electron-hole pair.
      We therefore have the normalization factor
\par
\begin{equation}
\label{eq:Particle:G0}
G_0 = \frac{2E}{\eta \tau \sqrt{\pi}} .
\end{equation}
\section{Fermi-Dirac Statistics}
\index{Fermi-Dirac distribution}In general, the electron and hole concentrations in semiconductors are defined by Fermi-Dirac
      distributions and density of states:
\par
\begin{subequations}
\begin{align}
 n = N_c {\cal F}_{1/2} \left(\eta _n\right) \\
 p = N_v {\cal F}_{1/2} \left(\eta _p\right)
\end{align}
\end{subequations}
The $\eta _n$ and $\eta _p$ are defined as
      follows:
\par
\begin{subequations}
\begin{align}
 \eta_n = \frac{ E_{F_n} - E_c }{k_b T} = {{\cal F}_{1/2}}^{-1} \left(\frac{n}{N_c}\right) \\
 \eta_p = \frac{ E_v - E_{F_p} }{k_b T} = {{\cal F}_{1/2}}^{-1} \left(\frac{p}{N_v}\right)
\end{align}
\end{subequations}
where $E_{F_n}$ and $E_{F_p}$ are the electron and
      hole Fermi energies. The relationship of Fermi energy and Fermi potential is
$E_{F_n}=-q\phi_n$, $E_{F_p}=-q\phi_p$.
\par
\marginhead{Evaluate Inverse Fermi Integral}${{\cal F}_{1/2}}^{-1}$
is the inverse Fermi integral of order one-half. Joyce
        and Dixon gives its approximation analytic expression in the year of 1977
\cite[Joyce1977]{}, which
        can be given by:
\begin{widetext}
\begin{equation}
{{\cal F}_{1/2}}^{-1} (x) =   \log(x) + ax + bx^2 + cx^3 + dx^4 x < 8.463
         \left[ \left( {\dfrac{3\sqrt \pi }{4} x} \right)^{3/4} - \dfrac{\pi^2}{6} \right]^{1/2}
\end{equation}
\end{widetext}
otherwise,
\begin{subequations}
\begin{align}
 a = 0.35355339059327379 \\
 b = 0.0049500897298752622 \\
 c = 1.4838577128872821 \times 10^{-4} \\
 d = 4.4256301190009895 \times 10^{-6}
\end{align}
\end{subequations}
In the GENIUS code, the $\eta _n$ and $\eta _p$ are
      derived from carrier concentration by Joyce-Dixon expression.
\par
For convenience, we introduce floowing two parameters as referred by
\cite[SEDAN1985]{}:
\par
\begin{subequations}
\begin{align}
 \gamma_n = \frac{ {\cal F}_{1/2} \left( \eta_n \right) } { \exp \left( \eta _n \right) }  \\
 \gamma_p = \frac{ {\cal F}_{1/2} \left( \eta_p \right) } { \exp \left( \eta _p \right) }
\end{align}
\end{subequations}
The carrier concentration for Fermi statistics and Boltzmann statistics can be described uniformly
      by:
\par
\begin{subequations}
\begin{align}
 n = N_c \gamma_n \exp \left( \eta _n \right) \\
 p = N_v \gamma_p \exp \left( \eta _p \right)
\end{align}
\end{subequations}
where $\gamma_n=\gamma_p=1$ for Boltzmann statistics, and less than
$1.0$ for Fermi statistics.
\par
\marginhead{DD Equation with Fermi Statistics}Consider the drift-diffusion current
\eqref{eq:Equation:DDML1:DDMCurrent}, p. \pageref{eq:Equation:DDML1:DDMCurrent}
when the carrier
        satisfies Fermi statistics and forces zero net current in equilibrium state, one can get the modified current
        equation, for which the Einstein relationship:
\par
\par
\begin{subequations}
\begin{align}
 D_n = \frac{k_b T}{q} \mu_n \\
 D_p = \frac{k_b T}{q} \mu_p
\end{align}
\end{subequations}
should be replaced by:
\par
\begin{subequations}
\begin{align}
 D_n = \frac{k_b T}{q} \mu_n {\cal F}_{1/2} \left( \eta_n \right)/{\cal F}_{-1/2} \left(
        \eta_n \right) \\
 D_p = \frac{k_b T}{q} \mu_p {\cal F}_{1/2} \left( \eta _p \right)/{\cal F}_{-1/2} \left(
        \eta_p \right)
\end{align}
\end{subequations}
where ${\cal F}_{-1/2}$ is the Fermi integral of order minus one-half. The
      corresponding current equation for electrons is
\par
\begin{equation}
\label{eq:Equation:FermiDirac:Jn1}
\vec{J}_n = \mu_n \left( qn \vec{E}_n + k_b T \lambda_n \nabla n \right)
\end{equation}
where
\par
\begin{equation}
\lambda_n = \frac{ {\cal F}_{1/2} \left(\eta _n\right) } { {\cal F}_{-1/2} \left(\eta _n\right)
        }
\end{equation}
The Fermi integral has an useful property:
\par
\begin{equation}
\frac{\partial}{\partial \eta}{\cal F}_\nu (\eta) = {\cal F}_{\nu - 1} (\eta)
\end{equation}
From the above property, one can derive two useful derivatives:
\par
\begin{subequations}
\begin{align}
 \frac{\partial }{\partial n} \eta_n (n) = \frac{\lambda_n}{n} \\
 \frac{\partial }{\partial n} \gamma_n (n) = \frac{ \gamma _n }{n} \left( {1 - \lambda_n } \right)
\end{align}
\end{subequations}
With the two derivatives, \eqref{eq:Equation:FermiDirac:Jn1}, p. \pageref{eq:Equation:FermiDirac:Jn1}
can be rewritten into the following
      equivalent formula:
\par
\begin{equation}
\vec{J}_n = \mu_n \left( qn \vec{E}_n + k_b T\nabla n - n k_b T\nabla \left( \ln \gamma _n \right)
        \right)
\end{equation}
The last term is the modification to Einstein relationship, which can be combinated into potential term. As
      a result, the current \eqref{eq:Equation:DDML1:DDMCurrent},
p. \pageref{eq:Equation:DDML1:DDMCurrent} keeps unchanged, but the effective driving
      force should be modified as:
\par
\begin{equation}
\vec{E}_n = \frac{1}{q}\nabla E_c - \frac{k_b T}{q}\nabla \left( \ln (N_c ) - \ln (T^{3/2} ) \right)
        - \frac{ k_b T}{q}{\nabla \left( \ln \gamma _n \right)}
\end{equation}
The same formula exists for holes:
\par
\begin{equation}
\vec{E}_p = \frac{1}{q}\nabla E_v + \frac{k_b T}{q}\nabla \left( \ln (N_v ) - \ln (T^{3/2} ) \right)
        + \frac{ k_b T}{q}{\nabla \left( \ln \gamma _p \right)}
\end{equation}
As a conclusion, when Fermi statistics is considered, the formula of DD method keeps unchanged, only an
      extra potential term should be considered. However, Fermi statistics also effect the implement of Ohmic boundary
      condition, please refer to [[TODO]]
\par
